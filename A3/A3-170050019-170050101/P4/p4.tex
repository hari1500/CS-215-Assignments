% !TeX spellcheck = en_GB
\documentclass[12pt]{article}

%opening
\title{Assignment 3 : CS215}
\author{Akshat Chugh : 170050019 \\ Satvik Ambati : 170050101}

\usepackage{amsmath}
\usepackage{amssymb}
\usepackage{hyperref}
\usepackage{ulem}
\usepackage{physics}
\usepackage{graphicx}
\graphicspath{./}
\usepackage[margin=0.5in]{geometry}


\begin{document}

\maketitle
\begin{enumerate}
	\setcounter{enumi}{3}
	\item 
	Assume our data is given by $x_1, x_2, \dots x_n$.
	
	Now, for $\hat{\theta}^{ML}$, 
	$$
	\text{Likelihood} = \begin{cases}
		\dfrac{1}{\theta^n}, & \text{if } \forall i,  \theta \geq x_i \\
		0, & \text{otherwise}
	\end{cases}$$
	
	Therefore, 
	$$
	\text{Likelihood} = \begin{cases}
	\dfrac{1}{\theta^n}, & \text{if } \theta \geq max(x_i) \\
	0, & \text{otherwise}
	\end{cases}$$
	
	Now, since the non-zero part is decreasing, $\hat{\theta}^{ML}$ = $max(x_i)$. 
	
	
	Now, for $\hat{\theta}^{MAP}$, our prior is,
	
	$$
	P(\theta) \propto \begin{cases}
	\left(\dfrac{\theta_m}{\theta}\right)^\alpha, & \text{if } \theta \geq \theta_m \\
	0, & \text{otherwise}
	\end{cases}$$
	 
	Thus, our posterior is,
	
	$$
	P(\theta | x_1, x_2, \dots x_n) \propto \begin{cases}
	\dfrac{\theta_m^\alpha}{\theta^{n+\alpha}}, & \text{if } \theta \geq max(\theta_m, max(x_i)) \\
	0, & \text{otherwise}
	\end{cases}$$
	
	Again, to maximise this, we just take $\hat{\theta}^{MAP} = max(\theta_m, max(x_i))$ as the non-zero part of the expression is decreasing.
	
	Now, if $\theta_m > \theta_{true}$, $\hat{\theta}^{MAP}$ will always be equal to $\theta_m$, no matter what the sample size. So, it is not necessary that $\hat{\theta}^{MAP}$ will tend to $\hat{\theta}^{ML}$ as $n$ tends to infinity. This is not desirable as we want our estimator to tend to ML estimator.
	\newpage
	
	Now, for $\hat{\theta}^{PM}$,
	\begin{align}
		\hat{\theta}^{PM} = E[\theta | Posterior] &= \frac{\displaystyle \int_{\theta_l}^{\infty} \dfrac{\theta_m^\alpha d\theta}{\theta^{n+\alpha-1}} }{\displaystyle \int_{\theta_l}^{\infty} \dfrac{\theta_m^\alpha d\theta}{\theta^{n+\alpha}}} & \text{(where } \theta_l = max(\theta_m, max(x_i)) \text{)}\\ \nonumber \\
		&= \dfrac{\dfrac{\theta_l^{-(n+\alpha-2)}}{n+\alpha-2}}{\dfrac{\theta_l^{-(n+\alpha-1)}}{n+\alpha-1}} & (\text{as } n + \alpha > 2) \\ \nonumber \\
		&= \theta_l(\frac{n+\alpha-1}{n+\alpha-2}) 
	\end{align}
	
	Thus, $\hat{\theta}^{PM} = \hat{\theta}^{MAP}(\frac{n+\alpha-1}{n+\alpha-2})$.
	
	Now, as $n$ tends to infinity, $\hat{\theta}^{PM}$ tends to $\hat{\theta}^{MAP}$. So it does not always tend to $\hat{\theta}^{ML}$. Thus, similarly, posterior mean estimator is also not desirable. 
	
	 
	
\end{enumerate}


\end{document}
